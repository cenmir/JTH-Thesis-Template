
\chapter*{Appendix 1: Lyx}

\addcontentsline{toc}{chapter}{Appendix 1: Lyx} 

\LyX{} (together with Mik\TeX{}) can be downloaded from www.lyx.org.
After you install \LyX{}, the first thing you need to do is to click
Help and read Tutorial. You will not be able to use \LyX{} without
that. Some hints: 
\begin{itemize}
\item Press Ctrl-R to make and view pdf. 
\item \LyX{} is based on the principle that ``What You See Is What You
\emph{Mean}.'' You type what you mean, and \LyX{} will take care
of typesetting it for you, so that the output looks nice. A \textsf{Return}
grammatically separates paragraphs, and a \textsf{Space} grammatically
separates words.
\item The \textsf{Environment} choice box is located on the left end of
the toolbar (the choice box below \textsf{``File Edit View}'').
It indicates in which environment you are currently writing. ``Standard''
is the default environment for text. Use ``Theorem'' to write statement
of a theorem, ``Proof'' for proof, etc. 
\item Use \textsf{Insert\lyxarrow{}Label} to label your theorems and \textsf{Insert\lyxarrow{}Cross~Reference}
to insert a reference to a particular theorem. The theorems, lemmas,
definitions, etc will be numbered automatically. Use \textsf{Insert\lyxarrow{}Citation}
to refer to an item in the Bibliography. Use Ctrl-M to enter Maths
mode and \textsf{Space} (or \textsf{Esc)} to leave the formula. Use
the arrow keys to navigate inside the formula. Use \_ to enter indices
(subscripts) and \textasciicircum{} for superscripts. You can also
use \LaTeX{} commands in Maths mode (e.g. \textbackslash{}sqrt, \textbackslash{}sin,
\textbackslash{}cup, etc).
\item Use Ctrl-L to enter \LaTeX{} code directly in the text if necessary
(it will appear in a red box). 
\end{itemize}
Example of a simple math formula: $a^{2}+b^{3}=\sin x+\sqrt{\alpha}$;
and with Maths Macros defined above: $F=\dlim F_{\alpha}$; $K=\Ker\varphi$;
$\myint ab{x^{2}}$.

Displayed math formula:

\[
\sum_{n=0}^{\infty}\Gamma_{n}x^{n}\ge\int_{a}^{b}\left(\frac{\gamma\cdot\omega(y)}{\lim_{x\to0}f(x)}\right)dy
\]


and a numbered one:

\begin{equation}
a^{2}+b^{2}=c^{2}\label{eq:pythagoras}
\end{equation}


Use Insert-> Cross Reference to insert a reference to it. Equation
(\ref{eq:pythagoras}) is widely known.
