%% LyX 2.1.3 created this file.  For more info, see http://www.lyx.org/.
%% Do not edit unless you really know what you are doing.
\documentclass[11pt,twoside,english]{scrbook}
\usepackage[T1]{fontenc}
\usepackage[latin9]{inputenc}
\usepackage[a4paper]{geometry}
\geometry{verbose,tmargin=30mm,bmargin=30mm,lmargin=30mm,rmargin=15mm}
\pagestyle{empty}
\usepackage{color}
\usepackage{babel}
\usepackage{verbatim}
\usepackage{float}
\usepackage{amsthm}
\usepackage{amsmath}
\usepackage{amssymb}
\usepackage{graphicx}
\usepackage{setspace}
\usepackage{esint}
\setstretch{1.3}
\usepackage[unicode=true,pdfusetitle,
 bookmarks=true,bookmarksnumbered=false,bookmarksopen=false,
 breaklinks=false,pdfborder={0 0 0},backref=false,colorlinks=false]
 {hyperref}

\makeatletter

%%%%%%%%%%%%%%%%%%%%%%%%%%%%%% LyX specific LaTeX commands.
\providecommand{\LyX}{\texorpdfstring%
  {L\kern-.1667em\lower.25em\hbox{Y}\kern-.125emX\@}
  {LyX}}
\DeclareRobustCommand*{\lyxarrow}{%
\@ifstar
{\leavevmode\,$\triangleleft$\,\allowbreak}
{\leavevmode\,$\triangleright$\,\allowbreak}}
\floatstyle{ruled}
\newfloat{algorithm}{tbp}{loa}[chapter]
\providecommand{\algorithmname}{Algorithm}
\floatname{algorithm}{\protect\algorithmname}

%%%%%%%%%%%%%%%%%%%%%%%%%%%%%% Textclass specific LaTeX commands.
\usepackage{geometry}
\usepackage{changepage}
\usepackage{txfonts}
\usepackage{scrlayer-scrpage}
\usepackage{jththesis} %Changes the Koma script style, adds commands
\usepackage{varprop} %Enables to set variables with properties, used for supplements
\numberwithin{equation}{section}
\numberwithin{figure}{section}
\numberwithin{table}{section}
\theoremstyle{plain}
\ifx\thechapter\undefined
\newtheorem{thm}{\protect\theoremname}
\else
\newtheorem{thm}{\protect\theoremname}[chapter]
\fi

%%%%%%%%%%%%%%%%%%%%%%%%%%%%%% User specified LaTeX commands.
\usepackage{lipsum}

\makeatother

\usepackage{listings}
\providecommand{\theoremname}{Theorem}
\renewcommand{\lstlistingname}{Listing}

\begin{document}
\definecolor{gray1}{RGB}{235, 235, 235} 
\definecolor{gray2}{RGB}{60, 60, 60}

\newgeometry
{ 
a4paper, 
top=30mm, 
bottom=30mm, 
inner=30mm, 
outer=15mm 
} 






\begin{titlepage}

\begin{center}
\vspace*{0.15\textheight plus 0.5\textheight minus 0.1\textheight }
\par\end{center}

\begin{center}
\textsf{LICENTIATE THESIS}
\par\end{center}

\begin{center}
\vspace{0.05\textheight}
\par\end{center}

\begin{spacing}{1.4}
\noindent \begin{center}
\textsf{\textbf{\huge{}PUT THE TITLE OF THE THESIS HERE WITH A SECOND
LINE IF IT IS VERY LONG}}
\par\end{center}{\huge \par}
\end{spacing}

\begin{center}
\vspace{0.05\textheight}
\par\end{center}

\begin{singlespace}
\noindent \begin{center}
\textsf{CANDIDATE'S FULL NAME}
\par\end{center}
\end{singlespace}

\begin{center}
\vspace{0.15\textheight}
\par\end{center}

\begin{center}
\includegraphics[width=0.4\paperwidth]{Graphics/newSchoolLogo}
\par\end{center}

\begin{center}
\vspace{0.50\textheight plus 0.5\textheight minus 0.50\textheight}
\par\end{center}

\begin{center}
\textsf{\textit{Department of Product Development}}\textsf{}\\
\textsf{SCHOOL OF ENGINEERING, J�NK�PING UNIVERSITY}\\
\textsf{J�nk�ping, Sweden 2015}
\par\end{center}

\end{titlepage}




\cleardoublepage{}

\pagenumbering{roman} 


\chapter*{Abstract}

\addcontentsline{toc}{chapter}{Abstract}

\noindent \noindent This is the abstract. A summary of the thesis in about 200
words. 

\noindent \lipsum[1-4]



\chapter*{Acknowledgements}

\addcontentsline{toc}{chapter}{Acknowledgements}

\noindent \input{Acknowledgements.tex}

\begin{comment}
Dedication

The dedication is usually quite short, and is a personal rather than
an academic recognition. The dedication should not be listed in the
table of contents. The chapter header is optional.
\end{comment}


\cleardoublepage{} % Comment if using header
% \chapter*{Dedication} % The clear double page is not needed if you use chapter*

\noindent \begin{center}
The dedication is usually quite short, and is a personal rather than
an academic recognition. The dedication should not be listed in the
table of contents. The chapter header is optional.
\par\end{center}


\chapter*{Statement of Co-Authorship}

\addcontentsline{toc}{chapter}{Statement of Co-Authorship}

\noindent \input{\string"Statement of Co-Authorship.tex\string"}

\noindent %
\begin{comment}

\chapter*{Declaration}

\addcontentsline{toc}{chapter}{Declaration}

All sentences or passages quoted in this project dissertation from
other people's work have been specifically acknowledged by clear cross
referencing to author, work and page(s). I understand that failure
to do this amounts to plagiarism and will be considered grounds for
failure in this module and the degree examination as a whole.

\bigskip{}


\noindent Name: 

\bigskip{}


\noindent Signed: 

\bigskip{}


\noindent Date: 
\end{comment}



\chapter*{Supplements}

\addcontentsline{toc}{chapter}{Supplements}

\noindent \setproperty{Suplement1}{number}
{1} 

\setproperty{Suplement1}{title}
{Author1, Author2, Author3; Really really long technical titel of the paper with a lot of words that are complex and descriptive to the reader to see the format} 

\setproperty{Suplement1}{footer}
{Candidate was the main author. Person 1 and person 2 contributed with advice regarding the work.}

\setproperty{Suplement2}{number}
{2} 

\setproperty{Suplement2}{title}
{Author1, Author2, Author3; Second paper title here, this one is shorter} 

\setproperty{Suplement2}{footer}
{Candidate was the main author. Person 1 contributed with advice regarding the work.}

\setproperty{Suplement3}{number}
{3} 

\setproperty{Suplement3}{title}
{Author1, Author2; Third paper title here} 

\setproperty{Suplement3}{footer}
{Candidate did the numerical implementations. Author 2 supplied the theoretical framework.}


\newcommand{\newSupplementFormat}[3]{%
\medskip{}
\noindent \rule[0.5ex]{1\columnwidth}{0.5pt}
\noindent \begin{minipage}[t]{0.25\textwidth}% 
\textbf{#1}% 
\end{minipage}%
\begin{minipage}[t]{0.75\textwidth}%
#2% 
\end{minipage}

\noindent \medskip{}

\noindent \textit{\scriptsize{}#3}{\scriptsize \par}
\vspace{8mm}
}

\noindent The following supplements constitute the basis of this thesis.

\newSupplementFormat
{Supplement I}
{\getproperty{Suplement1}{title}}
{\getproperty{Suplement1}{footer}}

\newSupplementFormat
{Supplement II}
{\getproperty{Suplement2}{title}}
{\getproperty{Suplement2}{footer}}

\newSupplementFormat
{Supplement III}
{\getproperty{Suplement3}{title}}
{\getproperty{Suplement3}{footer}}


\tableofcontents{}





\global\long\def\bbR{\mathbb{R}}
 

\global\long\def\ccR{\mathcal{R}}
 

\global\long\def\dlim{\operatorname{\underrightarrow{{\rm lim}}}}


\global\long\def\Ker{\operatorname{\rm Ker}}
 

\global\long\def\End{\operatorname{\rm End}}
 

\global\long\def\myint#1#2#3{\int_{#1}^{#2}\sin#3dx}
 


\chapter*{Introduction}

\addcontentsline{toc}{chapter}{Introduction}

\vspace{-1.2cm}


\noindent \pagenumbering{arabic} 

\noindent \noindent %\chapIntro{The background for the current work is described, followed by an outline of the theory behind the computational model.}


\chapIntro{The background for the current work is described, followed by an
outline of the theory behind the computational model.}

\noindent This is where arabic numbering starts

In this thesis we consider the work of Gauss \cite{Neumann2005} and
Hilbert \cite{Hilbert1893} on the subject of the title. 




\section*{Gauss's work}

We will discuss Gauss's work in Chapter \ref{chapter-on-Gauss}.




\section*{Hilbert's work}

We will discuss Hilbert's work in Chapter \ref{chapter-on-Hilbert}.


\noindent 

\noindent 
\chapter{Results of Gauss}

\chapIntro{Intro to chapter 1}

\label{chapter-on-Gauss}

In \cite{Neumann2005} Gauss proved the following very important result.
\begin{thm}
\cite[Theorem A]{Hilbert1893} \label{Gauss'sTheorem} Some very profound
result. 
\end{thm}
Later on in Chapter \ref{chapter-on-Hilbert} we will have more to
say about Theorem \ref{Gauss'sTheorem}.


\section{Gauss's youthful work}

\lipsum[1-30]


\section{Gauss's mature work}

\lipsum[31-60]




\noindent 

\noindent 
\chapter{Results of Hilbert}

\chapIntro{Intro to chapter 2}

\label{chapter-on-Hilbert}

In \cite{Hilbert1893} Hilbert considered these questions from a more
abstract point of view. He proved the following result.
\begin{thm}
\cite{Hilbert1893} \label{Hilbert'sTheorem} Some even more profound
result. 
\end{thm}
In Chapter \ref{chapter-on-Gauss} a special case of Theorem \ref{Hilbert'sTheorem}
was proved. We can prove an even more general result.
\begin{thm}
\label{mytheorem} An extremely profound result. \end{thm}
\begin{proof}
As any fool can plainly see, it's true! \end{proof}



\noindent 

\noindent 
\chapter{Algorithms}

This is the third chapter.

Exampel algorithm:

\begin{algorithm}[h]
\begin{lstlisting}[language=Matlab,numbers=left,numberstyle={\footnotesize},basicstyle={\ttfamily},breaklines=true,keywordstyle={\color{blue}}]
sq2=sqrt(2); knod=4; dof=3;
X=[TO.XC,TO.YC,TO.ZC]; xnod=TO.XC; ynod=TO.YC; znod=TO.ZC; 
nodes=TO.Connectivity; 
surfh = [SO.Surface.surfh]; 
nTetEle = size(nodes,1); 
nTriEle = size(surfh,2);
\end{lstlisting}
\protect\caption{Some algorithm}
\end{algorithm}



\noindent 

\noindent 
\chapter*{Appendix 1: Lyx}

\addcontentsline{toc}{chapter}{Appendix 1: Lyx} 

\LyX{} (together with Mik\TeX{}) can be downloaded from www.lyx.org.
After you install \LyX{}, the first thing you need to do is to click
Help and read Tutorial. You will not be able to use \LyX{} without
that. Some hints: 
\begin{itemize}
\item Press Ctrl-R to make and view pdf. 
\item \LyX{} is based on the principle that ``What You See Is What You
\emph{Mean}.'' You type what you mean, and \LyX{} will take care
of typesetting it for you, so that the output looks nice. A \textsf{Return}
grammatically separates paragraphs, and a \textsf{Space} grammatically
separates words.
\item The \textsf{Environment} choice box is located on the left end of
the toolbar (the choice box below \textsf{``File Edit View}'').
It indicates in which environment you are currently writing. ``Standard''
is the default environment for text. Use ``Theorem'' to write statement
of a theorem, ``Proof'' for proof, etc. 
\item Use \textsf{Insert\lyxarrow{}Label} to label your theorems and \textsf{Insert\lyxarrow{}Cross~Reference}
to insert a reference to a particular theorem. The theorems, lemmas,
definitions, etc will be numbered automatically. Use \textsf{Insert\lyxarrow{}Citation}
to refer to an item in the Bibliography. Use Ctrl-M to enter Maths
mode and \textsf{Space} (or \textsf{Esc)} to leave the formula. Use
the arrow keys to navigate inside the formula. Use \_ to enter indices
(subscripts) and \textasciicircum{} for superscripts. You can also
use \LaTeX{} commands in Maths mode (e.g. \textbackslash{}sqrt, \textbackslash{}sin,
\textbackslash{}cup, etc).
\item Use Ctrl-L to enter \LaTeX{} code directly in the text if necessary
(it will appear in a red box). 
\end{itemize}
Example of a simple math formula: $a^{2}+b^{3}=\sin x+\sqrt{\alpha}$;
and with Maths Macros defined above: $F=\dlim F_{\alpha}$; $K=\Ker\varphi$;
$\myint ab{x^{2}}$.

Displayed math formula:

\[
\sum_{n=0}^{\infty}\Gamma_{n}x^{n}\ge\int_{a}^{b}\left(\frac{\gamma\cdot\omega(y)}{\lim_{x\to0}f(x)}\right)dy
\]


and a numbered one:

\begin{equation}
a^{2}+b^{2}=c^{2}\label{eq:pythagoras}
\end{equation}


Use Insert-> Cross Reference to insert a reference to it. Equation
(\ref{eq:pythagoras}) is widely known.


\bibliographystyle{plain}
\phantomsection\addcontentsline{toc}{chapter}{\bibname}\bibliography{library}


\noindent 

\noindent \input{\string"Appended papers.tex\string"}
\end{document}
